\documentclass[A4paper, 12pt, oneside]{book}
\usepackage[left=1in, right=1in]{geometry}
\usepackage[czech]{babel}
\usepackage{inputenc}
\usepackage[T1]{fontenc}
\usepackage{graphicx}
\usepackage{setspace}
\usepackage{csquotes}
\usepackage{amsmath, amssymb, amsthm}
\usepackage[inline]{asymptote}
\usepackage{esdiff, icomma}
\usepackage{blindtext}

\usepackage[backend=biber, url=true]{biblatex}
\usepackage{url}
\addbibresource{soc.bib}

\newcommand{\B}[1]{\textbf{#1}}
\renewcommand{\S}[1]{\textsc{#1}}
\newcommand{\I}[1]{\textit{#1}}
\newcommand{\mB}[1]{\mathbf{#1}}
%\renewcommand{\vec}{\vec}
\newcommand{\ap}{{\,\prime}}
\newcommand{\abs}[1]{\lvert #1 \,\rvert}

%\newcommand{\vec}

\setstretch{1.3}
\renewcommand*\contentsname{Obsah}
\renewcommand{\theequation}{\arabic{equation}}

\begin{document}
\voffset = -10mm
\begin{center}
	{\LARGE \B{\S{Středoškolská odborná činnost}}} \\
	{\large \B{{Obor č. 2: Fyzika}}}
	\vfill
	{\Huge \B{Mechanika rodin planetek \\ s aplikací na rodinu Eunomia}}
\end{center}
	\vfill
{\large \bfseries Adam Křivka \\
	Jihomoravský kraj \hfill Brno 2018}

\newpage

TODO: Ostatní nutné úvodní stránky pro SOČku...

\newpage
\tableofcontents
\newpage
%\addcontentsline{toc}{section}{Úvod}

\chapter{Úvod do nebeské mechaniky}
TODO: Úvod
\vspace{40mm}
\section{Pohybové rovnice}
Pohybová rovnice je matematicky zapsaný fyzikální vztah, který popisuje možné pohyby těles v daném prostředí \autocite{wiki:eqm}. Řešením pohybové rovnice je funkce, popisující polohu a rychlost každého zkoumaného tělesa v závislosti na čase. Přitom potřebujeme znát počáteční podmínky --- polohy a rychlosti těles na začátku. Pohybová rovnice bývá ve tvaru diferenciální rovnice, což je rovnice, která vyjadřuje vztah mezi nějakou funkcí a jejími derivacemi, což je okamžitá změna hodnoty funkce při velmi malé změně argumentu, v našem případě času. 

V následující části se pokusíme nalézt řešení pohybové rovnice pro tělesa ve sluneční soustavě. Zákony, jimiž se budou naše tělesa řídit, jsou Newtonův gravitační zákon a Newtonovy pohybové zákony, které byly poprvé definovány již v roce 1687.
\subsection{Rovnice pro dvě tělesa}
Omezme se nyní na dvě tělesa a nalezněme řešení tzv. problému dvou těles, někdy také Keplerovy úlohy. To znamená, že se pokusíme odvodit funkci, popisující polohu a rychlost obou těles v závisloti na čase. 

Nacházíme se v inerciální vztažné soustavě, což je taková vztažná soustava, kde platí první Newtonův zákon. Jako bod v klidu si zvolme těžiště soustavy. Pro síly působící na obě tělesa podle Newtonova gravitačního zákona a druhého a třetího pohybového zákona platí
\begin{align} 
	\vec{F}_1 &= +G\frac{m_1m_2}{\abs{\vec{r}}^3}\vec{r} = m_1\vec{a}_1 \label{eq:newton1} \\
	\vec{F}_2 &= -G\frac{m_1m_2}{\abs{\vec{r}}^3}\vec{r} = m_2\vec{a}_2, \label{eq:newton2}
\end{align}
kde $G$ označuje gravitační konstantu, $m_1$, $m_2$ hmotnosti zkoumaných těles, $\vec{a}_1$, $\vec{a}_2$ vektory zrychlení těles (tj. druhé derivace polohových vektorů $\vec{r}_1$, $\vec{r}_2$ podle času) a $\vec{r}$ vektor udávající vzájemnou polohu těles, definovanou jako $\vec{r} = \vec{r}_2 - \vec{r}_1$. Součtem obou rovnic dostáváme
\begin{equation} \label{eq:mr=0}
	\vec{F}_1 + \vec{F}_2 = m_1\vec{a_1} + m_2\vec{a_2} = 0.
\end{equation}
Vektor popisující polohu těžiště soustavy je $\vec{R} \equiv \frac{m_1\vec{r}_1 + m_2\vec{r}_2}{m_1 + m_2}$. Jeho druhou derivací podle času dostáváme zrychlení
\begin{equation*}
	\diff[2]{\vec{R}}{t} = \frac{m_1\vec{a_1} + m_2\vec{a_2}}{m_1+m_2} = 0,
\end{equation*}
které se podle \eqref{eq:mr=0} rovná nule, takže se těžiště soustavy pohybuje konstantní rychlostí.
\newpage
Nyní se však přesuňme do soustavy neinerciální, kde je první z těles (běžně to hmotnější) nehybné. Řekněme, že nehybné těleso má index $1$, tedy nově $\vec{r}^\ap_1=0$, $\vec{r}^\ap_2=\vec{r}$ (tedy i $\vec{a}^\ap_2 = \vec{a}$) a $\vec{r}^\ap=\vec{r}$. Provedli jsme tedy v podstatě transformaci, kdy jsme ke každému vektoru přičetli $\vec{r}_1$.  Rovnici \eqref{eq:newton1} můžeme přepsat jako
\begin{align}
	\nonumber {\vec{a}} = Gm_2\frac{\vec{r}}{\abs{\vec{r}}^3} \\
		\diff[2]{\vec{r}}{t} - Gm_2\frac{\vec{r}}{\abs{\vec{r}}^3} = 0	
\end{align}
Často ještě definujeme gravitační paramter soustavy $\mu=Gm_2$.

I přesto, že tato diferenciální rovnice ještě není ve své konečné podobě vhodné k tomu, abychom z ní vyvodili následující vztah, prozradíme, že je jím funkce v polárních souřadnicích, popisující vzdálenost těles $r\equiv\abs{\vec{r}}$ v závisloti na úhlu $\theta$, který svírá přímka procházející oběma tělesy a nějaká zvolená referenční přímka. 
\begin{equation} \label{eq:polar}
	r(\theta)=\frac{p}{1+e\cos{(\theta-\omega)}}
\end{equation}
kde $p$ se nazývá paramter elipsy, jehož velikost je určena hodnotou $\mu$, $e$, resp. $\omega$ jsou integrační konstanty a nazývají se excentricita, resp. argument pericentra. K rovnici \eqref{eq:polar} a jejím důsledkům se vrátíme v sekci \ref{sec:orbelem}, zatím vězme, že jsme dostali obecnou funkci kuželosečky, z nichž nás bude nejvíce zajímat případ elipsy. Zmíněné konstanty budou určovat její tvar, rozhodně ale nestačí k úplnému popsání orientace trajektorie (orbity) tělesa v prostoru. 

Uvědomme si, že jsme neodvodili závislost polohy tělesa na čase. Tuto závislost určuje Keplerova rovnice:
\begin{equation} \label{eq:kepler}
M = E + e\sin E
\end{equation}
kde $M$ označuje střední anomálii, $E$ excentrickou anomálii a $e$ excentricitu elipsy (viz obrázek). Obě anomálie mají úhlové jednotky, úhel $M$ ale nemůžeme zkonstruovat, nicméně je významný tím, že je lineárně závislý na čase. Pokud známe $E$, můžeme pomocí snadno spočítat $M$.  Problém spočívá v tom, že nemůžeme vyjádřit $E$ v závisloti na $M$ konečným výrazem, ale pouze nekonečnou řadou nebo jej můžeme aproximovat numerickými metodami.

\subsection{Rovnice pro N těles}
Jak vidíme, už i pro problém dvou těles se musíme k získání polohy tělesa v čase uchýlit k aproximačním metodám. Ukáže se, že obecný problém $N$ těles je analyticky neřešitelný\footnote{pro omezený problém tří těles ale existují nějaká analytická řešení} a jediné aplikovatelné metody jsou metody numerické.

Uvažujme nyní $N$ těles -- idealizovaných bodových částic, které na sebe vzájemně gravitačně působí, v souladu s Newtonovým gravitačním zákonem. Pro jedno z těles, označené indexem $k\in\{1,2,...,N\}$, je celková síla $F_k$, která na něj působí, výslednicí všech gravitačních sil způsobených ostatními tělesy, jak ukazuje následující rovnice.
\begin{align} \label{eq:nbody}
		\vec{F_k} = m_k\vec{a_k} &= \sum_{\substack{i=1 \\ i\neq k}}^N G\frac{m_km_i}{(r_i-r_k)^3}(\vec{r_i}-\vec{r_k}) \nonumber \\
		\vec{a_k} &= \sum_{\substack{i=1 \\ i\neq k}}^N G\frac{m_i}{(r_i-r_k)^3}(\vec{r_i}-\vec{r_k}) 
\end{align}
kde $\vec{r_i-r_k}$ je vektor určující relativní pozici těles $i$ a $k$, konkrétně je to vektor s počátkem v tělese $k$ a vrcholem v tělese $i$; ostatní veličiny jsou definované analogicky jako v předchozí sekci.
\subsubsection{Eulerova metoda}
I přesto, že se tato metoda v reálných výpočtech téměř nepoužívá, uvádíme ji zde z didaktických důvodů, neboť názorně a relativně jednoduše ilustruje použití numerických metod pro řešení problému $N$ těles. Jak již název napovídá, poprvé s ní v 18. století přišel švýcarský matematik Leonhard Euler.

Princip algoritmu spočívá v tom, že v každém čase umíme z \eqref{eq:nbody} vypočítat zrychlení každého tělesa, podle toho odpovídajícím způsobem pozměnit vektor rychlosti. V následujícím kroku necháme všechna tělesa po dobu časového kroku pohybovat se po přímce s konstantní rychlostí. Následuje přesné zachycení algoritmu a k němu příslušná ilustrace na obrázku \ref{fig:euler}.

Mějme tedy zmiňovaných $N$ bodových částic splňujících \eqref{eq:nbody}, u kterých známe jejich hmotnost. Zaměřme se pouze na jednu částici a označme její počáteční polohu $\vec{r_0}$ a počáteční rychlost $\vec{v_0}$. (K použití Eulerovy metody potřebujeme ale znát i počáteční polohy i rychlosti všech ostatních těles v systému.) Dále určeme velikost časového kroku jako $h$.
\begin{enumerate}
	\item Nechť je poloha zvolené částice v čase $t_i$ $\vec{r_i}$ a rychlost $\vec{v_i}$. Protože známe i polohy a hmotnosti všech ostatních těles, můžeme z \eqref{eq:nbody} vypočítat zrychlení $\vec{a_i}$. 
	\item Položme $t_{i+1} = t_i+h$ a vypočítejme $\vec{v_{i+1}} = \vec{v_i} + h\vec{a_i}$.\footnote{Můžeme porovnat se vzorcem pro rovnoměrný přímočarý pohyb dobře známý ze středoškolského prostředí: $v = v_0 + at$, podobně v kroku 3 $s = s_0 + vt$}
	\item Vypočítejme $\vec{r_{i+1}} = \vec{r_i} + h\vec{v_i}$ a vraťme se ke kroku 1, tentokrát počítaje v čase $t_{i+1}$. 
\end{enumerate}

\begin{figure}
	\centering
	\begin{asy}
		unitsize(5cm);

		marker mark1 = marker(scale(circlescale*2)*unitcircle, Fill);
		marker mark2 = marker(scale(circlescale*7)*unitcircle, Fill);
		real ascale = 1;
		real vscale = 0.5;

		pair R = (0,0);
		pair r0 = (0.57,-1);
		pair M = 0.3;
		pair m = 1;
		pair G = 1;

		real h = 0.8;

		draw(R, marker=mark2);
		draw(r0, marker=mark1);
		label("$M$", shift(-0.05,-0.05)*R, SW);

		draw(arc(R,length(R-r0), -60, 15), longdashed+gray(0.7));

		// První iterace
		pair a0 = (G*M/(length(R-r0)**2))*unit(R-r0);
		draw(r0--shift(r0)*scale(ascale)*a0, arrow=EndArrow);
		label("$\mathbf{a_0}$", shift(r0)*scale(0.5)*scale(ascale)*a0, SSW);

		pair v0 = rotate(-90)*unit(a0)*sqrt(G*M/(length(R-r0)));
		draw(r0--shift(r0)*scale(vscale)*v0, arrow=EndArrow);
		label("$\mathbf{v_0}$", shift(r0)*scale(0.5)*scale(vscale)*v0, SE);

		pair v1 = v0+h*a0;
		draw(r0--shift(r0)*scale(vscale)*v1, arrow=EndArrow);
		label("$\mathbf{v_1}$", shift(r0)*scale(0.4)*scale(vscale)*v1, NNW); 

		pair r1 = r0 + h*v1;
		draw(r0--r1, dashed);
		draw(r1, marker=mark1);

		// Druhá iterace
		pair a1 = (G*M/(length(R-r1)**2))*unit(R-r1);
		draw(r1--shift(r1)*scale(ascale)*a1, arrow=EndArrow);
		label("$\mathbf{a_1}$", shift(r1)*scale(0.5)*scale(ascale)*a1, SSW);

		// pair v1
		draw(r1--shift(r1)*scale(vscale)*v1, arrow=EndArrow);
		label("$\mathbf{v_1}$", shift(r1)*scale(0.5)*scale(vscale)*v1, SE);

		pair v2 = v1+h*a1;
		draw(r1--shift(r1)*scale(vscale)*v2, arrow=EndArrow);
		label("$\mathbf{v_2}$", shift(r1)*scale(0.4)*scale(vscale)*v2, NW); 

		pair r2 = r1 + h*v2;
		draw(r1--r2, dashed);
		draw(r2, marker=mark1);

		// Třetí iterace
		pair a2 = (G*M/(length(R-r2)**2))*unit(R-r2);
		draw(r2--shift(r2)*scale(ascale)*a2, arrow=EndArrow);
		label("$\mathbf{a_2}$", shift(r2)*scale(0.5)*scale(ascale)*a2, SSW);

		// pair v2
		draw(r2--shift(r2)*scale(vscale)*v2, arrow=EndArrow);
		label("$\mathbf{v_2}$", shift(r2)*scale(0.5)*scale(vscale)*v2, SE);

		pair v3 = v2+h*a2;
		draw(r2--shift(r2)*scale(vscale)*v3, arrow=EndArrow);
		label("$\mathbf{v_3}$", shift(r2)*scale(0.4)*scale(vscale)*v3, NW); 

		pair r3 = r2 + h*v3;
		draw(r2--r3, dashed);
		draw(r3, marker=mark1);

		file fout = output("out.txt");
		write(fout, length(R-r0));
		//write(fout, length(v0));
	\end{asy}
	\caption{Ilustrace Eulerovy metody pro dvě tělesa, kdy větší těleso s pozicí $\vec{R}$ a hmotností $M$ (velká tečka vlevo nahoře) gravitačně působí na menší těleso (malé tečky vpravo). Jsou ukázány iterace $t_0$, kdy vycházíme z počátečních hodnot veličin $\vec{r_0}$ a $\vec{v_0}$, a $t_1$, $t_2$. Algoritmus byl reálně aplikován, se zjednodušenými hodnotami: $h=0,4 \ s$, $M=0,3 \ kg$, $G=1$, $r=R-r_0=1.15 \ m$, $v_0=0,51 \ ms^{-1}$ a obrázek byl mírně upraven pro lepší viditelnost. Šedá křivka naznačuje ideální dráhu tělesa.}
	\label{fig:euler}
\end{figure}

Jak můžeme vidět na obrázku \ref{fig:euler}, vypočtená dráha se od té reálné značně vzdaluje. To by samozřejmě vyřešila volba menší kroku $h$, ale pro velký počet těles a velkou požadovanou přesnost je algoritmus velmi pomalý (tedy konverguje pomalu) a dnešní počítače na něj nestačí.

TODO: O ostatních integračních metodách, udělám až budu víc chápat WHM, RMVS, ... Swift obecně.

\section{Orbitální elementy} \label{sec:orbelem}
TODO: Zavedení šesti základní elementů
\subsection{Oskulační elementy}
TODO: Popis, efemeridy
\subsection{Střední elementy}
\subsection{Vlastní elementy}
TODO: Význam, nastínění výpočtu

\chapter{Planetky ve Sluneční soustavě}
\section{Rodiny planetek}
\subsection{Metody identifikace rodin}
\subsubsection{Rezonance středního pohybu}
\subsubsection{Rezonance sekulární}

\chapter{Vlastnosti rodiny Eunomia}
\section{Nejistoty pozorovaných dat}
\section{Fyzikální model pro rodinu Eunomia}
\subsection{Jarkovského jev}
\subsection{YORP jev}
\subsection{Náhodné srážky}
\subsection{Nevratné děje při vývoji}
\section{Simulace orbitálního vývoje}
\section{Porovnání modelu a pozorování}

\chapter{Aplikace v praxi}
\chapter{Budoucí možnosti výzkumu}

\printbibliography
\end{document}
