\documentclass[A4paper, 12pt, oneside]{book}
\usepackage[left=1in, right=1in]{geometry}
\usepackage[czech]{babel}
\usepackage{inputenc}
\usepackage[T1]{fontenc}
\usepackage[pdftex]{graphicx}
\usepackage{enumitem}
\usepackage{epstopdf}
\usepackage{setspace}
\usepackage{csquotes}
\usepackage{amsmath, amssymb, amsthm}
\usepackage[inline]{asymptote}
\usepackage{esdiff, icomma, subcaption}
\usepackage{blindtext}

\usepackage[backend=biber, url=true, sorting=none]{biblatex}
\usepackage{url}
\addbibresource{soc.bib}

\newcommand{\B}[1]{\textbf{#1}}
\renewcommand{\S}[1]{\textsc{#1}}
\newcommand{\I}[1]{\textit{#1}}
\newcommand{\mB}[1]{\mathbf{#1}}
\newcommand{\ap}{{\,\prime}}
\newcommand{\abs}[1]{\lvert #1 \,\rvert}

\setstretch{1.3}
\renewcommand*\contentsname{Obsah}
\renewcommand{\theequation}{\arabic{equation}}

\begin{document}
\voffset = -10mm
\begin{center}
	{\LARGE \B{\S{Středoškolská odborná činnost}}} \\
	{\large \B{{Obor č. 2: Fyzika}}}
	\vfill
	{\Huge \B{Mechanika rodin planetek \\ s aplikací na rodinu Eunomia}}
\end{center}
	\vfill
{\large \bfseries Adam Křivka \\
	Jihomoravský kraj \hfill Brno 2018}

\newpage

TODO: Ostatní nutné úvodní stránky pro SOČku \ldots

\newpage
\tableofcontents
\newpage
%\addcontentsline{toc}{section}{Úvod}

\chapter{Úvod do nebeské mechaniky}
TODO: Úvod
\vspace{40mm}
\section{Pohybové rovnice}
Pohybová rovnice je matematicky zapsaný fyzikální vztah, který popisuje možné pohyby těles v daném prostředí \cite{wiki:eqm}. Řešením pohybové rovnice je funkce, popisující polohu a rychlost každého zkoumaného tělesa v závislosti na čase. Přitom potřebujeme znát počáteční podmínky --- polohy a rychlosti těles na začátku. Pohybová rovnice bývá ve tvaru diferenciální rovnice, což je rovnice, která vyjadřuje vztah mezi nějakou funkcí a jejími derivacemi, což je okamžitá změna hodnoty funkce při velmi malé změně argumentu, v našem případě času. 

V následující části se pokusíme nalézt řešení pohybové rovnice pro tělesa ve sluneční soustavě. Zákony, jimiž se budou naše tělesa řídit, jsou Newtonův gravitační zákon a Newtonovy pohybové zákony, které byly poprvé definovány již v roce 1687.
\subsection{Rovnice pro dvě tělesa} \label{sec:2body}
Omezme se nyní na dvě tělesa a nalezněme řešení tzv.\ problému dvou těles --- Keplerovy úlohy. To znamená, že se pokusíme odvodit funkci, popisující polohu a rychlost obou těles v závisloti na čase. 

Nacházíme se v inerciální vztažné soustavě, což je taková vztažná soustava, kde platí první Newtonův zákon. Jako bod v klidu si zvolme těžiště soustavy. Pro síly působící na obě tělesa podle Newtonova gravitačního zákona a druhého a třetího pohybového zákona platí
\begin{align} 
	\vec{F}_1 &= +G\frac{m_1m_2}{\abs{\vec{r}}^3}\vec{r} = m_1\vec{a}_1 \label{eq:newton1} \\
	\vec{F}_2 &= -G\frac{m_1m_2}{\abs{\vec{r}}^3}\vec{r} = m_2\vec{a}_2, \label{eq:newton2}
\end{align}
kde $G$ označuje gravitační konstantu, $m_1$, $m_2$ hmotnosti zkoumaných těles, $\vec{a}_1$, $\vec{a}_2$ vektory zrychlení těles (tj.\ druhé derivace polohových vektorů $\vec{r}_1$, $\vec{r}_2$ podle času) a $\vec{r}$ vektor udávající vzájemnou polohu těles, definovanou jako $\vec{r} = \vec{r}_2 - \vec{r}_1$. Součtem obou rovnic dostáváme
\begin{equation} \label{eq:mr0}
	\vec{F}_1 + \vec{F}_2 = m_1\vec{a_1} + m_2\vec{a_2} = 0.
\end{equation}
Vektor popisující polohu těžiště soustavy je $\vec{R} \equiv \frac{m_1\vec{r}_1 + m_2\vec{r}_2}{m_1 + m_2}$. Jeho druhou derivací podle času dostáváme zrychlení
\begin{equation*}
	\diff[2]{\vec{R}}{t} = \frac{m_1\vec{a_1} + m_2\vec{a_2}}{m_1+m_2} = 0,
\end{equation*}
které se podle \eqref{eq:mr0} rovná nule, takže se těžiště soustavy pohybuje konstantní rychlostí.
\newpage
Nyní se však přesuňme do soustavy neinerciální, kde je první z těles (běžně to hmotnější) nehybné. Řekněme, že nehybné těleso má index $1$, tedy nově $\vec{r}^\ap_1=0$, $\vec{r}^\ap_2=\vec{r}$ (tedy i $\vec{a}^\ap_2 = \vec{a}$) a $\vec{r}^\ap=\vec{r}$. Provedli jsme v podstatě transformaci, kdy jsme ke každému vektoru přičetli $\vec{r}_1$.  Rovnici \eqref{eq:newton1} můžeme přepsat jako
\begin{align}
	\nonumber {\vec{a}} = Gm_2\frac{\vec{r}}{\abs{\vec{r}}^3} \\
		\diff[2]{\vec{r}}{t} - Gm_2\frac{\vec{r}}{\abs{\vec{r}}^3} = 0	
\end{align}
Často ještě definujeme gravitační paramter soustavy $\mu=Gm_2$.

I přesto, že tato diferenciální rovnice ještě není ve své konečné podobě vhodné k tomu, abychom z ní vyvodili následující vztah, prozradíme, že je jím funkce v polárních souřadnicích, popisující vzdálenost těles $r\equiv\abs{\vec{r}}$ v závisloti na úhlu $\theta$, který svírá přímka procházející oběma tělesy a nějaká zvolená referenční přímka:
\begin{equation} \label{eq:polar}
	r(\theta)=\frac{p}{1+e\cos{(\theta-\omega)}}.
\end{equation}
Vztah \eqref{eq:polar} je obecným předpisem kuželosečky --- hyperboly, paraboly, elipsy nebo kružnice; pro naše účely se zaměřme na případ elipsy, kdy se v jednom z jejích ohnisek nachází centrální těleso.

$p$ označuje parametr elipsy, jehož velikost je určena hodnotou $\mu$ a pro který platí vztah
\begin{align}
	p=\frac{b^2}{a},
\end{align}
kde $a$ označuje délku hlavní poloosy, což je úsečka spojující střed elipsy s jedním z průsečíků elipsy s hlavní osou --- přímkou spojující ohniska, a $b$ označuje délku vedlejší poloosy, což je úsečka spojující střed elipsy s průsečíkem elipsy s přímkou kolmou na hlavní poloosu a procházející středem elipsy --- vedlejší osou (viz obrázek~\ref{fig:elip}).

Dále $e$, resp.\ $\omega$ jsou integrační konstanty a nazývají se excentricita, resp.\ argument pericentra. Pro excentricitu platí vztah
\begin{figure}[!htb] 
	\centering
	\begin{asy}
		size(8cm,8cm);

		marker mark1 = marker(scale(circlescale*2)*unitcircle, Fill);

		real a = 1.5;
		real b = 1;

		draw(ellipse((0,0), a, b));

		draw((-1.5*a,0)--(1.5*a,0));
		draw((0,-b)--(0,b));
		draw((0,0), mark1);

		draw((a,0), mark1);
		draw((-a,0), mark1);
		label("$P$", (a,0), SE);
		label("$A$", (-a,0), SW);

		real F = sqrt((a-b)*(a+b));
		draw((F,0), mark1);
		draw((-F,0), mark1);
		label("$F_1$", (F,0), S);
		label("$F_2$", (-F,0), S);

		draw(brace((0,0), (a,0)));
		label("$a$", (a/2,0.25), N);

		draw(brace((0,0), (0,b)));
		label("$b$", (-0.25, b/2), W);

		draw((0,0)--(1.3,-1.1), arrow=EndArrow);
		draw(arc((0,0), 0.5, -degrees(atan(1.1/1.3)), 0));
		label("$\omega$", (0.35,-0.25), N);
	\end{asy}
	\caption{Eliptická oběžná dráha vesmírného tělesa. $a$ označuje délku hlavní poloosy, $b$ délku vedlejší poloosy, $F_1$ a $F_2$ označují polohy ohnisek elipsy, přičemž centrální těleso se nachází v bodě $F_1$, a $P$, resp.\ $A$ označují pericentrum, resp.\ apocentrum oběžné dráhy, tedy bod nejnižší, resp.\ nejvyšší vzdálenosti od centrálního tělesa, $\omega$ označuje argument pericentra.} \label{fig:elip}
\end{figure}
\begin{align}
	e=\sqrt{1-\frac{a^2}{b^2}},
\end{align}
a volně řečeno udává, jak moc se elipsa liší od kružnice. Hodnota excentricity se pro eliptické dráhy nachází v intervalu $(0,1)$, kde krajními případy jsou $e=0$: dráha má tvar kružnice, a $e=1$: dráha má tvar paraboly.

Argument pericentra je úhel, který svírá hlavní osa s referenční přímkou. Platí pro něj vztah
\begin{align}
	\theta=\omega+f,
\end{align}
kde $f$ označuje pravou anomálii, což je úhel, který svírá hlavní osa s průvodičem tělesa (viz obrázek~\ref{fig:E}).

\begin{figure}[!htb] 
	\centering
	May-DO: Předělat do jednotného stylu obrázků.
	\includegraphics[width=0.5\textwidth]{obr/Eanomaly.png}
	\caption{Ilustrace vztahu mezi excentrickou anomálií $E$ a pravou anomálií $f$. $a$, resp.\ $b$ značí délku hlavní, resp.\ vedlejší poloosy, $P$ značí polohu tělesa na elipse, $C$ značí střed elipsy, $F$ značí ohnisko, ve kterém se nachází centrální těleso a $e$ značí excentricitu (vymazat??)} \label{fig:E}
\end{figure}

Uvědomme si, že jsme neodvodili závislost polohy tělesa na čase. Tuto závislost určuje Keplerova rovnice:
\begin{equation} \label{eq:kepler}
M = E + e\sin E
\end{equation}
kde $M$ označuje střední anomálii, $E$ excentrickou anomálii (viz obrázek~\ref{fig:E}) a $e$ excentricitu elipsy. Pro $E$ a pravou anomálií $f$ platí vztah
\begin{align} \label {eq:fE}
	\tan \frac{f}{2} = \sqrt{\frac{1+e}{1-e}}\tan \frac{E}{2}.
\end{align}

Anomálie mají úhlové jednotky, úhel $M$ ale nemůžeme zkonstruovat, nicméně je významný tím, že je lineárně závislý na čase, neboť je určen vztahem 
\begin{align} \label{eq:M}
	M=nt,
\end{align}
kde $n$ označuje střední denní pohyb, jinak řečeno průměrnou úhlovou rychlost. Pokud známe $E$, můžeme pomocí Keplerovy rovnice snadno spočítat $M$. Problém spočívá v tom, že nemůžeme vyjádřit $E$ v závisloti na $M$ konečným výrazem, ale pouze nekonečnou řadou nebo jej můžeme aproximovat iteračními nebo numerickými metodami.

(Uvést iterační metodu, náš kód v Pythonu???)

\subsection{Rovnice pro N těles}
Jak vidíme, už i pro dvě tělesa se musíme k získání polohy tělesa v čase uchýlit k numerickým metodám. Ukazuje se, že obecný problém $N$ těles je analyticky neřešitelný\footnote{existují ale nějaká zajímavá speciální řešení, viz \cite{cohan12}.} a jediné aplikovatelné metody jsou metody přibližné analytické nebo numerické.

Uvažujme nyní $N$ těles --- respektive hmotných bodů, které na sebe vzájemně gravitačně působí v souladu s Newtonovým gravitačním zákonem. Pro libovolné těleso, označené indexem $i\in\{1,\,2,\,\dots,\,N\}$, je celková síla $F_i$, která na něj působí, výslednicí všech gravitačních sil způsobených ostatními tělesy, jak ukazuje následující rovnice.
\begin{align} 
	\vec{F}_i = m_i\vec{a}_i &= \sum_{\substack{j=1 \\ j\neq i}}^N G\frac{m_im_j}{\abs{\vec{r}_i-\vec{r}_j}^3}(\vec{r_i}-\vec{r_j}) \label{eq:nbody1}\\
		\vec{a}_i &= \sum_{\substack{j=1 \\ j\neq i}}^N \frac{Gm_j}{\abs{\vec{r}_i-\vec{r}_j}^3}(\vec{r_i}-\vec{r_j}) \label{eq:nbody2}
\end{align}
kde $\vec{r}_i-\vec{r}_j$ označuje vektor určující vzájemnou polohu těles $i$ a $j$, konkrétně jde o vektor s počátkem v tělese $j$ a vrcholem v tělese $i$; ostatní veličiny jsou definované analogicky jako v předchozí části.
\subsubsection{Eulerova metoda}
I přesto, že se následující integrační metoda v přesných numerických výpočtech zřídka používá, uvádíme ji zde z didaktických důvodů, neboť názorně ilustruje použití numerických metod pro řešení problému $N$ těles. Jak název napovídá, poprvé s ní v 18. století přišel švýcarský matematik Leonhard Euler.

Princip algoritmu spočívá v tom, že v libovolném čase můžeme z \eqref{eq:nbody2} vypočítat zrychlení každého tělesa. Pak, po zvolení určitého časového kroku, odpovídajícím způsobem změníme vektor rychlosti. Následně necháme všechna tělesa po dobu časového kroku pohybovat se po přímce konstantní rychlostí. Existují dvě verze Eulerovy metody, dopředná a zpětná, které se liší volbou rychlosti, se kterou necháváme pohybovat se po přímce, viz následující přesný popis obou metod a obrázek~\ref{fig:euler}.

Mějme zmiňovaných $N$ hmotných bodů, pro které platí \eqref{eq:nbody2}. Zaměřme se na jeden z nich a označme jeho počáteční polohu $\vec{r}(t_0)$ a počáteční rychlost $\vec{v}(t_0)$. K použití Eulerovy metody potřebujeme znát i počáteční polohy a rychlosti všech ostatních těles v systému. Dále vhodně zvolme velikost časového kroku $h$. V následujících třech krocích si ukážeme jednu iteraci algoritmu jak pro dopřednou, tak pro zpětnou metodu.

\begin{figure}[!htb] 
	\centering 
	\begin{subfigure}[b]{0.45\textwidth}
	\begin{asy}
		size(8cm,8cm);

		marker mark1 = marker(scale(circlescale*2)*unitcircle, Fill);
		marker mark2 = marker(scale(circlescale*3)*unitcircle, Fill);
		pen pen1 = linetype(new real[] {1,6})+linewidth(0.4);

		real au = 149597870700;

		real ascale = 3.2*au*pow10(1);
		real vscale = 0.8*au*pow10(-5);

		pair R = (0,0);
		pair r0 = (3/5*au,-4/5*au);
		pair m1 = 2*pow10(30);
		pair G = 6.67*pow10(-11);

		real h = 20*24*60*60;

		draw((-1/5*au,-5/5*au)--(7/5*au,-5/5*au)--(7/5*au,2/5*au)--(-1/5*au,2/5*au)--cycle, invisible);

		draw(R, marker=mark2);
		draw(r0, marker=mark1);
		label("$m_1$", shift(-0.05,-0.05)*R, SW);

		draw(arc(R,length(R-r0), -53, 15), longdashed+gray(0.7));

		// První iterace
		draw(R--r0, arrow=EndArrow, pen1);
		label("$\vec{r}_0$", shift(R)*scale(0.5)*r0,SW);

		pair a0 = (G*m1/(length(R-r0)**2))*unit(R-r0);
		draw(r0--shift(r0)*scale(ascale)*a0, arrow=EndArrow);
		label("$\vec{a}_0$", shift(r0)*scale(0.5)*scale(ascale)*a0, SSW);

		pair v0 = rotate(-90)*unit(a0)*sqrt(G*m1/(length(R-r0)));
		draw(r0--shift(r0)*scale(vscale)*v0, arrow=EndArrow);
		label("$\vec{v}_0$", shift(r0)*scale(0.5)*scale(vscale)*v0, SE);

		pair v1 = v0+h*a0;
		draw(r0--shift(r0)*scale(vscale)*v1, arrow=EndArrow);
		label("$\vec{v}_1$", shift(r0)*scale(0.4)*scale(vscale)*v1, NNW); 

		pair r1 = r0 + h*v1;
		draw(r0--r1, dashed);
		draw(r1, marker=mark1);

		// Druhá iterace
		draw(R--r1, arrow=EndArrow, pen1);
		label("$\vec{r}_1$", shift(R)*scale(0.5)*r1,SW);

		pair a1 = (G*m1/(length(R-r1)**2))*unit(R-r1);
		draw(r1--shift(r1)*scale(ascale)*a1, arrow=EndArrow);
		label("$\vec{a}_1$", shift(r1)*scale(0.5)*scale(ascale)*a1, SSW);

		// pair v1
		draw(r1--shift(r1)*scale(vscale)*v1, arrow=EndArrow);
		label("$\vec{v}_1$", shift(r1)*scale(0.5)*scale(vscale)*v1, SE);

		pair v2 = v1+h*a1;
		draw(r1--shift(r1)*scale(vscale)*v2, arrow=EndArrow);
		label("$\vec{v}_2$", shift(r1)*scale(0.4)*scale(vscale)*v2, NW); 

		pair r2 = r1 + h*v2;
		draw(r1--r2, dashed);
		draw(r2, marker=mark1);

		// Třetí iterace
		draw(R--r2, arrow=EndArrow, pen1);
		label("$\vec{r}_2$", shift(R)*scale(0.5)*r2,SW);

		pair a2 = (G*m1/(length(R-r2)**2))*unit(R-r2);
		draw(r2--shift(r2)*scale(ascale)*a2, arrow=EndArrow);
		label("$\vec{a}_2$", shift(r2)*scale(0.5)*scale(ascale)*a2, SSW);

		// pair v2
		draw(r2--shift(r2)*scale(vscale)*v2, arrow=EndArrow);
		label("$\vec{v}_2$", shift(r2)*scale(0.5)*scale(vscale)*v2, SE);

		pair v3 = v2+h*a2;
		draw(r2--shift(r2)*scale(vscale)*v3, arrow=EndArrow);
		label("$\vec{v}_3$", shift(r2)*scale(0.4)*scale(vscale)*v3, NW); 

		pair r3 = r2 + h*v3;
		draw(r2--r3, dashed);
		draw(r3, marker=mark1);

		draw(R--r3, arrow=EndArrow, pen1);
		label("$\vec{r}_3$", shift(R)*scale(0.5)*r3,S);

		//file fout = output("out.txt");
		//write(fout, length(R-r0));
		//write(fout, length(v0));
	\end{asy}
	\end{subfigure}
	\begin{subfigure}[b]{0.45\textwidth}
	\begin{asy}
		size(8cm,8cm);

		marker mark1 = marker(scale(circlescale*2)*unitcircle, Fill);
		marker mark2 = marker(scale(circlescale*3)*unitcircle, Fill);
		pen pen1 = linetype(new real[] {1,6})+linewidth(0.4);

		real au = 149597870700;

		real ascale = 3.2*au*pow10(1);
		real vscale = 0.8*au*pow10(-5);

		pair R = (0,0);
		pair r0 = (3/5*au,-4/5*au);
		pair m1 = 2*pow10(30);
		pair G = 6.67*pow10(-11);

		real h = 23.5*24*60*60;

		draw((-1/5*au,-5/5*au)--(7/5*au,-5/5*au)--(7/5*au,2/5*au)--(-1/5*au,2/5*au)--cycle, invisible);
		
		draw(R, marker=mark2);
		draw(r0, marker=mark1);
		label("$m_1$", shift(-0.05,-0.05)*R, SW);

		draw(arc(R,length(R-r0), -53, 15), longdashed+gray(0.7));

		// První iterace
		draw(R--r0, arrow=EndArrow, pen1);
		label("$\vec{r}_0$", shift(R)*scale(0.5)*r0,SW);

		pair a0 = (G*m1/(length(R-r0)**2))*unit(R-r0);
		draw(r0--shift(r0)*scale(ascale)*a0, arrow=EndArrow);
		label("$\vec{a}_0$", shift(r0)*scale(0.5)*scale(ascale)*a0, SW);

		pair v0 = rotate(-90)*unit(a0)*sqrt(G*m1/(length(R-r0)));
		draw(r0--shift(r0)*scale(vscale)*v0, arrow=EndArrow);
		label("$\vec{v}_0$", shift(r0)*scale(0.5)*scale(vscale)*v0, SE);

		pair v1 = v0+h*a0;
		draw(r0--shift(r0)*scale(vscale)*v1, arrow=EndArrow);
		label("$\vec{v}_1$", shift(r0)*scale(0.4)*scale(vscale)*v1, NNW); 

		pair r1 = r0 + h*v0;
		draw(r0--r1, dashed);
		draw(r1, marker=mark1);

		// Druhá iterace
		draw(R--r1, arrow=EndArrow, pen1);
		label("$\vec{r}_1$", shift(R)*scale(0.5)*r1,SW);

		pair a1 = (G*m1/(length(R-r1)**2))*unit(R-r1);
		draw(r1--shift(r1)*scale(ascale)*a1, arrow=EndArrow);
		label("$\vec{a}_1$", shift(r1)*scale(0.5)*scale(ascale)*a1, SW);

		// pair v1
		draw(r1--shift(r1)*scale(vscale)*v1, arrow=EndArrow);
		label("$\vec{v}_1$", shift(r1)*scale(0.5)*scale(vscale)*v1, SE);

		pair v2 = v1+h*a1;
		draw(r1--shift(r1)*scale(vscale)*v2, arrow=EndArrow);
		label("$\vec{v}_2$", shift(r1)*scale(0.4)*scale(vscale)*v2, NW); 

		pair r2 = r1 + h*v1;
		draw(r1--r2, dashed);
		draw(r2, marker=mark1);

		// Třetí iterace
		draw(R--r2, arrow=EndArrow, pen1);
		label("$\vec{r}_2$", shift(R)*scale(0.5)*r2,SW);

		pair a2 = (G*m1/(length(R-r2)**2))*unit(R-r2);
		draw(r2--shift(r2)*scale(ascale)*a2, arrow=EndArrow);
		label("$\vec{a}_2$", shift(r2)*scale(0.5)*scale(ascale)*a2, SW);

		// pair v2
		draw(r2--shift(r2)*scale(vscale)*v2, arrow=EndArrow);
		label("$\vec{v}_2$", shift(r2)*scale(0.5)*scale(vscale)*v2, SE);

		pair v3 = v2+h*a2;
		draw(r2--shift(r2)*scale(vscale)*v3, arrow=EndArrow);
		label("$\vec{v}_3$", shift(r2)*scale(0.4)*scale(vscale)*v3, NW); 

		pair r3 = r2 + h*v2;
		draw(r2--r3, dashed);
		draw(r3, marker=mark1);

		draw(R--r3, arrow=EndArrow, pen1);
		label("$\vec{r}_3$", shift(R)*scale(0.5)*r3,S);

		//file fout = output("out.txt");
		//write(fout, length(R-r0));
		//write(fout, length(v0));
	\end{asy}
	\end{subfigure}
	\caption{Ilustrace dopředné (vpravo) a zpětné (vlevo) Eulerovy metody pro dvě tělesa, kdy větší těleso (velká tečka vlevo nahoře) gravitačně působí na menší těleso (malé tečky vpravo). Jsou ukázány první tři iterace. Algoritmus byl doopravdy implementován, s hodnotami: $h=20\,{\rm \text{dnů}}$, $m_1=2\cdot10^{30}\,{\rm kg}$, $G=6,67\cdot10^{-11}\,{\rm m^3kg^{-1}s^{-2}}$, $|\vec{r}|=1\,{\rm AU}$, $v_0=29861\,{\rm ms^{-1}}$. Vektory jsou přeškálované. Šedá křivka znázorňuje analytické řešení problému dvou těles.} \label{fig:euler}
\end{figure}

\begin{enumerate}
	\item Nechť je v čase $t_k$ poloha zvoleného bodu $\vec{r}(t_k)$ a rychlost $\vec{v}(t_k)$. Z \eqref{eq:nbody2} vypočítáme zrychlení $\vec{a}(t_k)$. 
	\item Položme $t_{k+1} = t_{k}+h$ a vypočítejme $\vec{v}(t_{k+1}) = \vec{v}(t_k) + h\cdot\vec{a}(t_k)$.\footnote{Můžeme porovnat se vzorcem pro rovnoměrný přímočarý pohyb, dobře známým ze středoškolského učiva: $v = v_0 + at$, podobně v kroku $3$: $s = s_0 + vt$}
	\item Pro dopřednou metodu vypočítejme $\vec{r}(t_{k+1})$ jako $\vec{r}(t_{k+1}) = \vec{r}(t_k) + h\cdot\vec{v}(t_k)$ a pro zpětnou jako $\vec{r}(t_{k+1}) = \vec{r}(t_k) + h\cdot\vec{v}(t_{k+1})$. Poté se vraťme ke kroku $1$, tentokrát počítaje v čase $t_{k+1}$. 
\end{enumerate}

Jak můžeme vidět na obrázku~\ref{fig:euler}, vypočtená dráha se od té analytické značně vzdaluje. To by samozřejmě řešila volba menší kroku $h$, ale pro velký počet těles a velkou požadovanou přesnost je algoritmus velmi pomalý.

Jedno z možných vylepšení je volně řečeno průměrování dopředné a zpětné Eulerovy metody --- tzv.\ \enquote{leapfrog} metoda. Spočívá v tom, že rychlost počítáme v jedné polovině časového kroku, ne na konci nebo na začátku. Další zpřesnění lze získat tak, že místo pohybu po přímce konstantní rychlostí použijeme lokální eliptickou dráhu, kterou získáme, když zanedbáme všechna ostatní tělesa a uvážíme pouze centrální těleso. Tato integrační metoda se již podobá algoritmu Wisdom--Holman Mapping, jehož ještě zlepšenou verzi využívá integrační balíček SWIFT \cite{levison}, který budeme v této práci používat. Nutno dodat, že v námi užitém algoritmu ještě započítáváme negravitační jevy, jako Yarkovského jev, YORP jev a náhodné srážky, viz \cite{broz11}.

\section{Orbitální elementy} \label{sec:orbelem}
K úspěšnému určení a zařazení oběžné dráhy nějakého vesmírného tělesa zavedeme šest elementů dráhy, které budeme v pozdějších sekcích používat k analýze rodin planetek. V sekci~\ref{sec:2body} jsme odvodili obecnou rovnici kuželosečky zapsanou v polárních souřadnicích. Ve sluneční soustavě se však s jinými, než s eliptickými dráhami nesetkáme, budeme tedy definovat elementy dráhy pouze pro dráhu eliptickou.
\subsection{Oskulační elementy}
Oskulační elementy popisují takovou oběžnou dráhu tělesa, po které by se pohybovalo kolem centrálního tělesa v problému dvou těles --- tedy po zanedbání všech ostatních těles (planet, měsíců, \ldots) a negravitačních sil. Svým způsobem tedy zachycují aktuální stav tělesa v rámci celé soustavy, je tudíž nutno s nimi uvádět i časový údaj --- tzv.\ epochu. Neustále se mění působením perturbací, což jsou jakékoli vnější síly působící na těleso jiné, než gravitační síla centrálního tělesa --- např. gravitace ostatních planet, nerovnoměrný tvar centrálního tělesa či Jarkovského efekt, neboli unášení, o kterém budeme hovořit později.

Prvními dvěma elementy jsou hlavní poloosa a excentricita, které určují tvar elipsy (viz obrázek~\ref{fig:elip}). Hlavní poloosu značíme $a$ a při studiu sluneční soustavy tento údaj většinou udáváme v astronomických jednotkách --- AU, přičemž $1\, {\rm AU} = 149\,597\,870\,{\rm km}$, což je střední vzdálenost slunce a Země. 

Dalšími dvěma elementy jsou argument pericentra a střední anomálie (viz~\ref{sec:2body}), které udávají polohu tělesa v rovině oběžné dráhy. Referenčí přímkou je průsečnice roviny dráhy s refereční rovinou --- ekliptikou, přesněji řečeno je to polopřímka s počátečním bodem v poloze centrálního tělesa a pomocným bodem ve vzestupné uzlu, což je bod, ve kterém těleso prochází refereční rovinou \enquote{zespodu nahoru}. Střední anomálie je určená vztahem~\eqref{eq:M} a udává samotnou polohu tělesa na elipse.

Poslední dvojice elementů, sklon a délka vzestupného uzlu, udává polohu roviny oběžné dráhy v prostoru. Sklon dráhy (cizím slovem inklinace) je orientovaný úhel, který svírá rovina dráhy vzhledem k ekliptice. Znaší se $i$ a většinou se udává ve stupních, někdy se ale uvádí $\sin i$, což je ekvivalentní definice, protože pro $-90^o\leq i \leq 90^o$ je $\sin i$ jednoznačně určen. Délka vzestupného uzlu je orientovný úhel, který svírá spojnice centrálního tělesa s vzestupným uzlem s referenčím směrem v rovině ekliptiky, za který se ve sluneční soustavě bere směr k jarnímu bodu, což je jeden z průsečíků ekliptiky s rovinou zemského rovníku, jinak řečeno poloha slunce vzhledem k Zemi v okamžiku jarní rovnodennosti.

\subsubsection{Výpočet polohy tělesa z elementů dráhy}
Skutečnost, že elementů je právě šest, není náhodou, existuje totiž výpočet, kterým lze z polohy a rychlosti tělesa v prostoru, tedy z údajů $x,\, y,\, z,\, v_x,\, v_y,\, v_z$, vypočítat elementy dráhy v prostoru; je tedy logické, že vzniklých údajů musí být zase šest. 

Ukažme, pro účely této práce, jak z šesti elementů dráhy vypočítat polohu tělesa $x,\, y,\, z$ vzhledem k centrálnímu tělesu.

\begin{enumerate}[label=\arabic*.]
	\item Z rovnice \eqref{eq:kepler} nějakou ze jmenovanných metod (aproximačních, iteračních nebo numerických) vypočítáme velikost excentrické anomálie.
	\item Vztah \eqref{eq:fE} upravíme a spočteme pravou anomálii $f$:
		\begin{align}
			f = 2\tan^{-1}(\sqrt{\frac{1+e}{1-e}}\tan \frac{E}{2})
		\end{align}
	\item Pomocí vztahu 
		\begin{align}
			r=a(1-e\cos E)
		\end{align}
		vypočítáme velikost r --- relativní vzdálenost těles.
	\item Pomocí vztahů
		\begin{align}
			x&=r(\cos\Omega\cos(\omega+f)-\sin\Omega\sin(\omega+f)\cos i) \\
			y&=r(\sin\Omega\cos(\omega+f)+\cos\Omega\sin(\omega+f)\cos i) \\
			z&=r\sin i\sin(\omega+f)
		\end{align}
		vypočítáme $x,\,y,\,z$.
\end{enumerate}

(Uvést kód z Pythonu???)

\begin{figure}[!htb]
	\centering
	\includegraphics[width=0.8\textwidth]{obr/atOF}
	\caption{Porovnání oskulační a střední hlavní poloosy pro simluaci jedné planetky po dobu $3,76$ miliónů let.}
	\label{atOF}
\end{figure}

\subsection{Střední elementy}
Střední elementy jsou elementy dráhy zbavené krátkých periodických perturbací, jako jsou pohyby velkých planet, např. Jupitera nebo Saturnu. Pro jejich výpočet z oskulačních elementů lze použít analytické, numerické nebo také filtrační metody, které jsme v naší práci využili. 

Střední elementy odstraňují vliv tzv.\ rezonancí středního pohybu, což jsou oblasti vesmírného prostoru, ve kterém když se planetka nachází, tvoří poměr její periody s periodou nějaké jiné planety zlomek s nízkým čitatelem a jmenovatelem (viz~\ref{meanmotion}). 
\subsection{Vlastní elementy}

\begin{figure}[!htb]
	\centering
	\includegraphics[width=0.8\textwidth]{obr/atFP}
	\caption{Porovnání střední a vlastní hlavní poloosy pro simluaci jedné planetky po dobu $3,76$ miliónů let. Lze vidět, že se za tuto dobu vlastní hlavní poloosa planetky zvýšila o $0.00022554\,{\rm AU}=33740.3\, {\rm km}$. Tento jev vysvětluje Jarkovského efekt, viz~\ref{sec:jarko}.}
	\label{atFP}
\end{figure}

Vlastní elementy jsou elementy dráhy zbavené jak krátkých, tak dlouhých periodických perturbací, mezi které kromě již zmíněných patří navíc sekulární rezonance, které jsou způsobeném závislotí frekvencí precese (změny) argumentu perihélia a délky vzestupného uzlu planetky a nějaké jiné planety. 

Vlastní elementy jsou tedy svým způsobem \enquote{průměry} pohybu a jsou téměř neměnné ve velkém časovém úseku, ačkoliv působením negravitačních sil --- hlavně Jarkovského jevu --- se můžou pomale zvětšovat nebo zmenšovat. 

Mezi vlastní elementy počítáme pouze vlastní hlavní poloosu --- $a_p$, vlastní excentricitu --- $e_p$ a vlastní inklinaci --- $i_p$. Ostatní elementy nemá cenu uvažovat, protože argument perihélia i délka vzestupného uzlu periodicky precedují a střední anomálie je lineárně závislá na čase.

\chapter{Planetky ve sluneční soustavě}
\begin{figure}[!htb]
	\centering
	\includegraphics[width=0.6\textwidth]{obr/ssb.png}
	\caption{Zařazení těles sluneční soustavy podle IAU (po valném shromáždění v Praze roku 2006, kde byl definován pojem planeta, čímž bylo vyřazeno Pluto z planet sluneční soustavy.) Zdroj: \cite{wiki:ssb} \label{fig:ssb}}
\end{figure}

Podle Mezinárodní astronomické unie (IAU) se tělesa ve sluneční soustavě dělí hlavně na planety, komety, přirozené satelity a planetky. Planeta je definována jako takové těleso, které obíhá kolem Slunce, má dostatečnou hmotnost, aby si udrželo kulovitý tvar, a je \enquote{vyčistilo} svoje okolí od ostatních těles. Kometa je těleso složené z ledu a prachu, které většinou obíhá Slunce po velmi excentrické dráze a zanechává přitom za sebou ohon, který je způsoben vypařováním ledu z komety působením slunečního záření. Dále přirozené satelity jsou tělesa obíhající nějakou planetu nebo planetku. 

Konečně všechno ostatní je klasifikování jako planetka (někdy nepřesně označováno jako asteroid) --- patří sem trpasličí planety (těch je zatím pouze pět: Ceres, Pluto, Eris, Makemake, Haumea), transneptunická tělesa (tělesa, která se pohybují za oběžnou dráhou Neptunu), malá tělesa sluneční soustavy a další. Tyto skupiny se navzájem překrývají, viz \ref{fig:ssb}. Dále můžeme planetky dělit na podle jejich oběžné dráhy na tělesa vnitřní a vnější sluneční soustavy. Většina těles vnější sluneční soustavy se nachází v Kupierově pásu, který se sahá od oběžné dráhy Neptunu až do vzdálenosti přibližně $55\, {\rm AU}$. Tělesa vnitřní sluneční soustavy jsou přibližně omezená oběžnou dráhou Jupitera --- největší populace se nachází v hlavním pásu, který se nachází mezi oběžnonou dráhou Marsu (přbližně $1,5\, {\rm AU}$) a Jupitera (přibližně $5,2\, {\rm AU}$). Dále sem spadají také planetky, jejichž oběžná dráhá protínu dráhu Marsu nebo Země --- tzv. blízkozemní planetky, a planetky nacházející se v libračních centrech L4 a L5\footnote{To jsou takové body, v nichž se vyrovnává působení gravitační a odstředivé síly. Jde tedy o jakýsi bod rovnováhy. Body L4 a L5 se nachází na oběžné dráze většího tělesa o $60^o$ \enquote{napřed} nebo \enquote{za} tělesem.} soustav Slunce--Země, Slunce--Mars, Slunce--Jupiter --- tzv. trojáni. Hildy???

\begin{figure}[!htb]
	\centering
	\includegraphics[width=0.8\textwidth]{obr/mainbelt.png}
	\caption{Planetky hlavního pásu podle vlastních elementů --- osa $x$ znázorňuje vlastní poloosu a osa $y$ vlastní sklon. Lze vidět některé rodiny planetek, konkrétně například uprostřed nahoře Eunomia. Barevné označení znázorňuje oblasti mezi rezonancemi středního pohybu.} \label{fig:belt}
\end{figure}

Při našem studiu se budeme zaměřovat na planetky hlavního pásu, ve kterém se také nachází jediná trpasličí planeta ve vnitřní sluneční soustavě Ceres, který má střední poloměr $473\, {\rm km}$. Ostatní planetky mají střední poloměr od $250\, {\rm km}$ po pouze několik metrů???. Největší populace se rovnoměrně rozprostírá od vzdálenosti $2,1\, {\rm AU}$ od Slunce do vzdálenosti přibližně $3,3\, {\rm AU}$. Planetky, které vystoupí z této oblasti, se buď příblíží Marsu a jsou jeho gravitací vymrštěny na zcela odlišnou oběžnou dráhu, nebo se podobně přiblíží Jupiteru, který ji též může odklonit od původní oběžné dráhy nebo ji může zachytit ve svém gravitačním působení, čímž se planetka stane přirozenou družicí Jupitera. 

\section{Rezonance}
Struktura hlavního pásu planetek je významně ovlivněna rezonancemi, které nastávají, když nějaký údaj o dráze planetky a nějaké planety, běžně Jupitera nebu Saturnu, je v jednoduchém poměru, tzn. ve zlomku s nízkým čitatelem a jmenovatelem. Pro sekulárí rezonance můžeme mít i složitější vztahy, kdy spolu porovnáváme více veličin zároveň. ??????
\subsection{Rezonance středního pohybu}
Nástin výpočtu jejich polohy (je to jednoduché), vysvětlení vlivu na HCM

\subsection{Rezonance sekulární} 
Nějaký hezký obrázek (třeba ze složky secres???), dlouhodobý vliv


\section{Rodiny planetek}
Definice --- katastrofická srážka, rozptýlení ve střední anomálii, podobnost pouze ve vlastních elementech, přesunout sem kapitoly o Jarkovského jevu, YORPu a náhodných srážkách???
\subsection{Metody identifikace rodin}
Popis HCM (hierarchická shlukovací metoda), $v_{cutoff}$, 

\chapter{Vlastnosti rodiny Eunomia}
Postup určování rodiny, volba $v_{cutoff}$, pozadí --- [česky???] interlopers (ref na později), rozdělení velikostí 
\begin{figure}[!htb]
	\centering
	\includegraphics[width=0.6\textwidth]{obr/Nv}
	\caption{Závislost počtu členů rodiny Eunomia na zvolené hraniční rychlost $v_{cutoff}$. Počet členů prudce vzroste při přechodu z rychlosti $43\,{\rm m/s}$ na $44\,{\rm m/s}$, což je způsobené vzdáleností prvního tělesa jiného než mateřského tělesa. Dále vzroste prudce při přechodu z rychlosti $46\,{\rm m/s}$ na $47\,{\rm m/s}$, což je způsobené splynutím s jinou rodinou.}
	\label{Nv}
\end{figure}
\begin{figure}[!htb]
	\centering
	\begin{subfigure}[b]{0.45\textwidth}
	\includegraphics[width=\textwidth]{obr/size_distribution}
	\end{subfigure}
	\begin{subfigure}[b]{0.45\textwidth}
	\includegraphics[width=\textwidth]{obr/size_distribution_SMALLD}
	\end{subfigure}
	\caption{Histogram četnosti velikostí planetek, kde veličina $N(>D)$ označuje počet planetek s průměrem větším než $D$.???}
	\label{size_distribution}
\end{figure}
\section{Fyzikální model pro rodinu Eunomia}
Rozdělení v $ae$ a $ai$ prostoru, vliv rezonancí J8/3 a J13/5, Gaussovy rovnice --- elipsa, volba bodu rozpadu ($f=90^o$, $\omega+f=50^o$).
\begin{figure}[!htb]
	\centering
	\includegraphics[width=0.7\textwidth]{obr/ae_wise}
	\includegraphics[width=0.7\textwidth]{obr/ai_wise}
	\caption{Pozorovaná rodina Eunomia v rovině vlastní hlavní poloosy $a_p$ a vlastní excentricity $e_p$ a v rovině vlastní hlavní poloosy $a_p$ a vlastní inklinace $\sin i_p$. Barvy jsou převzaty z katalogu WISE. J8/3 a J13/5 označují rezonance středního pohybu s Jupiterem. Šedé křivky naznačují výpočet Gaussových rovnic pro hodnoty $f=0^o,\,90^o,\,180^o$ a $\omega+f=0^o,\, 50^o,\, 90^o$, kde nejlepší hodnotou je $f=90^o$ a $\omega+f=50^o$.}
	\label{ae_ai_wise}
\end{figure}
\subsubsection{[česky] Interlopers}
Odstranění interlopers, barevné indexy, SLOAN, WISE, ref na Nejistoty veřejných dat.
\begin{figure}[!htb]
	\centering
	\includegraphics[width=0.5\textwidth]{obr/astar_iz}
	\caption{Barevné indexy z katalogu SLOAN (ref článek o tom). Pro vyřazení interlopers byly zvoleny krajní hodnoty $0\leq a^* \leq 0.3$ a $-0.3\leq i-z \leq 0.3$.}
	\label{astar_iz}
\end{figure}
\begin{figure}[!htb]
	\centering
	\includegraphics[width=0.5\textwidth]{obr/pV_pIR}
	\caption{Barevné indexy z katalogu SLOAN (ref článek o tom). Pro vyřazení interlopers byly zvoleny krajní hodnoty $0.05 \leq p_v \leq 0.4$.}
	\label{pV_pIR}
\end{figure}
\subsection{Nevratné děje při vývoji}
Disipační síly, vliv na vývoj v prostoru vlastních elementů, reference na nějaký článek o nové databázi tvarů planetek???
\subsubsection{Jarkovského jev} \label{sec:jarko}
Popis, vysvětelní obrázku aH, vliv na vlastní elementy, účinky kolem rezonancí
\begin{figure}[!htb]
	\centering
	\includegraphics[width=0.5\textwidth]{obr/aH_wise}
	\caption{Rozdělení pozorované rodiny Eunomia v rovině vlastní hlavní poloosy $a_p$ a absolutní hvězdné velikosti $H$. Lze pozorovat typický tvar $V$, který je způsobem počátečním rozpadem a vlivem Jarkovského jevu, který je navíc ještě zesílen vlivem YORPu, což způsobuje větší koncentraci malých planetek na okrajích rodiny.}
	\label{aH_wise}
\end{figure}
\subsubsection{YORP jev}
Popis, zmínka o Poynting–Robertson efektu???, zesílení Jarkovského jevu
\subsubsection{Náhodné srážky}
???
\section{Nejistoty pozorovaných dat}
???

Následující dvě kapitoly budou napsány až po doběhnutí simulace. 
\section{(Simulace orbitálního vývoje)}
Popis experimentu (údaje o výpočetní technice)
\section{(Porovnání modelu a pozorování)}

\chapter{Aplikace v praxi} 
K čemu se to hodí: pochopení dějů ve sluneční soustavě, informace o jejím vzniku, katalog rodin planetek nám může říct něco o jejich složení --- těžba asteroidů
\chapter{Budoucí možnosti výzkumu}
???
\printbibliography
\end{document}
