\documentclass[a4paper, 12pt]{article}
\usepackage[czech]{babel}
\usepackage[utf8]{inputenc}
\usepackage[margin=0.6in]{geometry}
\usepackage{graphicx, enumitem}
\usepackage{amsmath, amssymb, amsthm}

\begin{document}
\Large
\begin{enumerate}[wide, label=\arabic*/21]
\item 
\uv{Vážená \textbf{poroto}, milí \textbf{posluchači}, rád bych Vám představil svoji \textbf{středoškolskou odbornou činnost}. Jmenuji se... studuji... a moje práce nese název... mým konzultantem...}
\item Nejprve...
\item první planetka \textbf{Ceres}, v našich katalozích půl milionu planetek, \textbf{průkopník} Kiyotsugu Hirayama, překlad anglického \textbf{minor planets}, dynamická struktura, migrace planet, \textbf{větší pravděpodobnost vzniku rodin}
\item vznik z \textbf{planetesimálů}, princip vzniku rodin
\item hlavní pás planetek mezi \textit{Marsem} a \textit{Jupiterem}, odkázat se na \textbf{vlastní elementy}, rozprostření po dráze
\item vysvětlit \textbf{Jarkovského jev}, ostatní podle času vynechat 
\item velmi stručně vyjmenovat \textbf{velkou poloosu}, \textbf{excentricitu} a \textbf{sklon}
\item vlivem \textbf{perturbací} se elementy dráhy mění, simulace jedné částice, lze pozorovat Jarkovského jev
\item rychlá ilustrace principu numerických integrátorů, \textbf{jedna částice je jednoduchá}, zdůraznit \textbf{náročnost problému}
\item nejprve musíme rodinu identifikovat -- \textbf{hierarchická shlukovací metoda}, z pozorovaných dat z \textbf{katalogů}

\

\hrule
\hfill \textsc{5min}
\newpage
\item (vyřazení přimísených těles) 
\item (vliv Jarkovského jevu)
\item vytvoření \textbf{syntetické populace}, zohlednění \textbf{rozdělení velikostí}, \textbf{přimíchání pozadí}, balíček SWIFT, délka bloku *
\item popsat \textbf{úvodní izotropní rozpad}, zmínit \textbf{orbitální rezonance} a \textbf{mechanismus opouštění rodiny}
\item rozdělení do \textbf{boxů}, šmitec
\item nejprve popsat \textbf{pouze pozadí}, \textbf{kompaktnost} jádra, vývoj v oblasti \textbf{mezi rezonance} 
\item znovu kompaktnost jádra, \textbf{vliv okolních rodin}, odstranění \textbf{Adeona}
\item \textbf{krátkodobé simulace}, (např. k ukázce střední a vlastních elementů), \textbf{znovu}identifikace, úspěšná simulace, struktura/\textbf{mechanismy}, odhad \textbf{stáří}
\item vývoj chí kvadrátu, eventuelně minimum --- odhad stáří, \textbf{publikace v prestižním vědeckém časopisu}, např. \textbf{Icarus}
\item \uv{Kromě dvou obrázků byly \textbf{všechny moje}, doporučená literatura, úvodní obrázek z první knihy}
\item \uv{\textbf{Ještě předtím}, než Vám poděkuji za pozornost bych chtěl \textbf{poděkovat svému konzultantovi} docentu \textit{Miroslavu Brožovi} za \textbf{vedení mé práce} a \textbf{poskytnutí zázemí} na \textit{Astronomickém ústavu univerzity Karlovy v Praze} a své škole \textit{Cyrilometodějskému gymnáziu}, zejména mé paní profesorce matematiky \textit{Veronice Svobodové}, za \textbf{podporu} při psaní práce.

Tím je to ode mě vše a já Vám děkuji za pozornost.}
\end{enumerate}
\end{document}
